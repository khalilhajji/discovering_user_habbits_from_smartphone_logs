%%%%%%%%%%%%%%%%%%%%%%%%%%%%%%%%%%%%%%%%%
% Masters/Doctoral Thesis 
% LaTeX Template
% Version 1.43 (17/5/14)
%
% This template has been downloaded from:
% http://www.LaTeXTemplates.com
%
% Original authors:
% Steven Gunn 
% http://users.ecs.soton.ac.uk/srg/softwaretools/document/templates/
% and
% Sunil Patel
% http://www.sunilpatel.co.uk/thesis-template/
%
% License:
% CC BY-NC-SA 3.0 (http://creativecommons.org/licenses/by-nc-sa/3.0/)
%
% Note:
% Make sure to edit document variables in the Thesis.cls file
%
%%%%%%%%%%%%%%%%%%%%%%%%%%%%%%%%%%%%%%%%%

%----------------------------------------------------------------------------------------
%	PACKAGES AND OTHER DOCUMENT CONFIGURATIONS
%----------------------------------------------------------------------------------------

\documentclass[11pt, oneside]{Thesis} % The default font size and one-sided printing (no margin offsets)

\graphicspath{{Pictures/}} % Specifies the directory where pictures are stored

\usepackage[square, numbers, comma, sort&compress]{natbib} % Use the natbib reference package - read up on this to edit the reference style; if you want text (e.g. Smith et al., 2012) for the in-text references (instead of numbers), remove 'numbers' 
\hypersetup{urlcolor=blue, colorlinks=true} % Colors hyperlinks in blue - change to black if annoying
\title{\ttitle} % Defines the thesis title - don't touch this

\begin{document}

\frontmatter % Use roman page numbering style (i, ii, iii, iv...) for the pre-content pages

\setstretch{1.3} % Line spacing of 1.3

% Define the page headers using the FancyHdr package and set up for one-sided printing
\fancyhead{} % Clears all page headers and footers
\rhead{\thepage} % Sets the right side header to show the page number
\lhead{} % Clears the left side page header

\pagestyle{fancy} % Finally, use the "fancy" page style to implement the FancyHdr headers

\newcommand{\HRule}{\rule{\linewidth}{0.5mm}} % New command to make the lines in the title page

% PDF meta-data
\hypersetup{pdftitle={\ttitle}}
\hypersetup{pdfsubject=\subjectname}
\hypersetup{pdfauthor=\authornames}
\hypersetup{pdfkeywords=\keywordnames}

%----------------------------------------------------------------------------------------
%	TITLE PAGE
%----------------------------------------------------------------------------------------

\begin{titlepage}
\begin{center}

\textsc{\LARGE \univname}\\[1.5cm] % University name
\textsc{\Large MasterThesis}\\[0.5cm] % Thesis type

\HRule \\[0.4cm] % Horizontal line
{\huge \bfseries \ttitle}\\[0.4cm] % Thesis title
\HRule \\[1.5cm] % Horizontal line
 
\begin{minipage}{0.4\textwidth}
\begin{flushleft} \large
\emph{Author:}\\
{\authornames} % Author name - remove the \href bracket to remove the link
\end{flushleft}
\end{minipage}
\begin{minipage}{0.4\textwidth}
\begin{flushright} \large
\emph{Supervisor:} \\
{\supname} % Supervisor name - remove the \href bracket to remove the link  
\end{flushright}
\end{minipage}\\[3cm]
 
\large \textit{A thesis submitted in fulfilment of the requirements\\ for the degree of \degreename}\\[0.3cm] % University requirement text
\textit{in the}\\[0.4cm]
\groupname\\\deptname\\[2cm] % Research group name and department name
 
{\large \today}\\[4cm] % Date
%\includegraphics{Logo} % University/department logo - uncomment to place it
 
\vfill
\end{center}

\end{titlepage}


%----------------------------------------------------------------------------------------
%	ABSTRACT PAGE
%----------------------------------------------------------------------------------------

\addtotoc{Abstract} % Add the "Abstract" page entry to the Contents

\abstract{\addtocontents{toc}{\vspace{1em}} % Add a gap in the Contents, for aesthetics

The Thesis Abstract is written here (and usually kept to just this page). The page is kept centered vertically so can expand into the blank space above the title too\ldots
}

\clearpage % Start a new page

%----------------------------------------------------------------------------------------
%	ACKNOWLEDGEMENTS
%----------------------------------------------------------------------------------------

\setstretch{1.3} % Reset the line-spacing to 1.3 for body text (if it has changed)

\acknowledgements{\addtocontents{toc}{\vspace{1em}} % Add a gap in the Contents, for aesthetics

The acknowledgements and the people to thank go here, don't forget to include your project advisor\ldots
}
\clearpage % Start a new page

%----------------------------------------------------------------------------------------
%	LIST OF CONTENTS/FIGURES/TABLES PAGES
%----------------------------------------------------------------------------------------

\pagestyle{fancy} % The page style headers have been "empty" all this time, now use the "fancy" headers as defined before to bring them back

\lhead{\emph{Contents}} % Set the left side page header to "Contents"
\tableofcontents % Write out the Table of Contents


\lhead{\emph{List of Figures}} % Set the left side page header to "List of Figures"
\listoffigures % Write out the List of Figures

\lhead{\emph{List of Tables}} % Set the left side page header to "List of Tables"
\listoftables % Write out the List of Tables


%----------------------------------------------------------------------------------------
%	SYMBOLS
%----------------------------------------------------------------------------------------

\clearpage % Start a new page

\lhead{\emph{Symbols}} % Set the left side page header to "Symbols"

\listofnomenclature{lll} % Include a list of Symbols (a three column table)
{
$a$ & distance & m \\
$P$ & power & W (Js$^{-1}$) \\
% Symbol & Name & Unit \\

& & \\ % Gap to separate the Roman symbols from the Greek

$\omega$ & angular frequency & rads$^{-1}$ \\
% Symbol & Name & Unit \\
}

%----------------------------------------------------------------------------------------
%	DEDICATION
%----------------------------------------------------------------------------------------

\setstretch{1.3} % Return the line spacing back to 1.3

\pagestyle{empty} % Page style needs to be empty for this page

\dedicatory{For/Dedicated to/To my\ldots} % Dedication text

\addtocontents{toc}{\vspace{2em}} % Add a gap in the Contents, for aesthetics

%----------------------------------------------------------------------------------------
%	THESIS CONTENT - CHAPTERS
%----------------------------------------------------------------------------------------

\mainmatter % Begin numeric (1,2,3...) page numbering

\pagestyle{fancy} % Return the page headers back to the "fancy" style

% Include the chapters of the thesis as separate files from the Chapters folder
% Uncomment the lines as you write the chapters

% Chapter 1

\chapter{Introduction} % Main chapter title

\label{Chapter1} % For referencing the chapter elsewhere, use \ref{Chapter1} 

\lhead{Chapter 1 \emph{Introduction}} % This is for the header on each page - perhaps a shortened title

%----------------------------------------------------------------------------------------


With the rapid and large deployment of the internet, the emergence of the cloud and the entrance to the internet of things area, the recent years were (and are still) marked by the exponential increase of the data stored and available of any kind. Nowadays, data streams for daily life; from computers, credit cards, TVs, trains, sensor equipped buildings and factories. The availability of this huge quantity of data is changing the way companies lead their business and are changing the rules of competitiveness. In a publication that zooms in the challenges and the power of big data, Mc Kinsey affirms that "the use of big data will become a key basis of competition and growth for individual firms" and they estimate that a retailer using big data to the full has the potential to increase its operating margin by more than 60 percent.\par

In this work, we are interested in the data collected from a gadget that shares the life of the user; his smartphone. Smartphone data contains the locations a user visits, the activities he does, the notification he receives, the applications he launches and many other information. This kind of data is unique in the sense that it represents a complete snapshot of a user's life. In a world driven by the power of data and the ability to anticipate and understand the needs of users, companies shows a big interest in studying this emerging kind of dataset. In the other hand, the richness  and the diversity of this data attracts the curiosity of researchers.\par

In this context, many work have been done with this kind of logs, many paths have been explored and multiple questions answered. Those researches are discussed later with more details.
In this work, we tackle an important question that escaped to the interest of previous researches: Having the smartphone logs of one user, can we find a model that exhibits his particular behaviors and habits? More practically, let's imagine the following example. Let's imagine that Bob has some particular habits: he does running while listening music on Saturdays, he visits his parents on Sundays, he reads news when he is in his office in the mornings, and he puts an alarm clock at 7am during his working days. The question we are answering in this work is the following: Can we find a model that discovers those particular behaviors by analyzing Bob smartphone's logs? How precise can this model be in doing this task?
It is important to note that we are interested in discovering the individual behaviors of a user, and thus we aim to find a method that takes as input the logs of a unique user. \par

In the recent few years, key innovations allowed smartphones to drastically evolve from cell-phone devices used for calling to powerful "pocket-computers" devices that can be used as cell-phone, camera, calendar, clock, game consol, web browser and many other roles at the same time. As an important company in the smartphone industry, Sony is one of the actors of the smartphones evolution and is aiming on keeping innovating this sector. 
\\Our work is a part of this continuous research for innovation. Indeed, it aims to allow smartphones building a personal relationship with their owner by adapting to their specific needs and answering heir specific requests. Let's keep the parallel with Bob to understand what does building a personal relationship with a user concretely means. Let's suppose that Bob's smartphone is able to learn the specific habits of Bob. When Bob forgets to put his alarm clock on a working day, his smartphone can remind him to do it. When an unusual traffic congestion appears in Sunday in the rode that Bob use to take to reach his parents place, his smartphone can inform him. Finally, when his smartphone does not have enough power to play music on Saturday morning, Bob's smartphone can remind him to recharge it because he will probably need it for running. Our work is a start in making it possible for a smartphone to adapt to it's owner's behaviors and to react interactively in some contexts. In other words, it is a start is making smartphones behaves smarter. \par

Scientifically speaking, this problem can be seen as a clustering problem: our goal is to find different clusters of data points where each cluster represents a particular behavior of the user. Coming back to Bob, running while listening to music on Saturdays morning can be represented by a cluster that contains the running activity, the application launch music, the day Saturday and the time frame 8am-12am. Putting an alarm clock at 7 pm during the working days can be represented as a cluster containing all the days of the week, the notification alarm and the time frame 7pm-8am. 
\\Clustering is a widely addressed problem in machine learning and data analysis, and it has been applied to many contexts and topics. It as been used for example in corpus-text modeling, recommender systems and image recognition. Different approaches have been developed to answer those challenges; the probabilistic latent topic modeling and the matrix factorization are examples of these different approaches. A parallel between our methods and these approaches is made multiple times in this thesis and it will be shown that it is of a strong benefit. 
\\Our problem sits at the interface between an emerging area of research that takes profit of the existence of a new kind of data and an area that constitutes one of the basis of the emergence of the machine learning and data analysis techniques. From a scientifically point fo view, addressing the clustering problem in a new emerging context makes our problem particularly challenging. \par

The thesis is organized as follows: In chapter 2, we introduce some notations and definitions, state the problem in a mathematical way and go through the researches done in this field. 
\\In chapter 3, we describe in details the Generative Hidden Class Model for Mixed Data Types (GHCM-MDT). It is model that we developed specifically to answer our needs and that shows to perform better than the other existing methods.
\\In chapter 4, we introduce other known and widely used models that has been shown to perform very well in doing tasks similar to ours. We use those models as baselines to evaluate the performances of GHCM-MDT. To have a complete overview, we both use some models based on the matrix factorization approach using some sophisticated techniques and others based on advanced methods of probabilistic latent topic modeling.
\\In chapter 5, we detail the metrics used to test the performance of the different models in performing the task needed.
\\In chapter 6, we present the results obtained with the different models and compare the performances of the different models to GHCM-MDT.
\\Finally, chapter 7 presents our conclusions.

%----------------------------------------------------------------------------------------
% Chapter 1

\chapter{Preliminaries} % Main chapter title

\label{Chapter2} % For referencing the chapter elsewhere, use \ref{Chapter1} 

\lhead{Chapter 2. \emph{Preliminaries}} % This is for the header on each page - perhaps a shortened title

%----------------------------------------------------------------------------------------

\section{Definitions and notations}

In this work, we take an interest on datasets containing smartphone log of a use. We refer to those datasets as $data$, $smartphone logs$, $user logs$ or $dataset$.

The particularity of those logs is that they contain different information sources and types. They contain for example information about the location of the user, the activities he is doing, the applications he uses, the settings he puts in the phone, the notifications he receives, the bluetooth devices he connects to and the external devices that he connects to the phone (headset, powerplug).
We refer to these different types as $features$ or $types$.

Smartphone logs contains multiple features ($Location$, $Notification$) where each feature can take multiple values. For example, feature $Location$ can take values $Home$, $Work$, the feature $Activity$ can take values as $on\_foot$, $on\_bicycle$, $in\_vehicle$ and $still$. We refer to the values of a feature $frame$ as the $values$ of feature $frame$ or the $vocabulary$ of feature $frame$.

We note the set $F= \left \{ 1,.., J \right \}$ as the set representing a user logs containing J different features. Here, we refer to features as ids and no longer as names. We nite $f \in F$ as the feature number $f$.

Similarly, we note the set $V_f = \left \{ 1,.., I_f \right \} $ as the set representing the $I_f$ different values that can be taken by $f, f \in F$. We refer to values as ids and no longer as names. Note that $I_f$ represents the vocabulary size of feature $f$. $v \in V_f$ indicates the value number $v$ of the feature $f$.

Thus, the complete values space of the dataset is completely defined by the sets $F$ and $v_1,...,v_f$. We refer to this $complete values space$ as the $language$ defined by the dataset. Note that the size of the language is just the sum of the vocabulary sizes of all the features ($\sum_{f=1}^{F} I_f$).

A $data point$, also called a $realization$ is then represented as the pair $(f,v)$ which means the $v^{th}$ value of the $f^{th}$ feature.

The user logs is nothing than a set of multiple realizations where each data point $(f,v)$ is linked with a time stamp indicating it s time of occurence.

Let the $record$ be the set of realizations that occured during a certain time frame. Thus a record containing n realizations $\left \{ (y_1,w_1),..., (y_N,w_N) \right \}$ that all occured in the same time frame can be represented as vector $\mathbf{r} = [(y_1,w_1),..., (y_N,w_N)]$. Here $y \in F$ denotes the feature of the $n^{th}$ realization and $w_n$ the values of the $n^{th}$ realization (i.e the value taken by the feature $y_n$).

Using the record definition, smartphone logs can be represented as a set of records where each record contains the data points that occured during a certain period in the time. A period could be for example 1 hour. In this case, each record represents 1 hour of the period of observation of a user.

Using this representation a smartphone logs can be represented as a set $R = \left \{ \mathbf{r_1},...,\mathbf{r_M} \right \} $, of $M$ records, where each record $\mathbf{r_m}$ has a size $N_m$, for $m \in \left \{ (1,...,M \right \}$.

We call $R$ a corpus and when considering about this representation, we may refer to smartphone logs as a $corpus$.

%----------------------------------------------------------------------------------------
\section{Problem Statement}
The goal that drives our work is to extract the behaviors and habits of a user from his smartphone logs data.
In fact, all individuals have their own behaviors and habbits that they repeat periodically over the time. For example, an individual generally use to work during the week days, to sleep at home during the night. He might also be playing sport once a week, listening to music while driving or reading news when he is in the bus.

The question is, without previously knowing any behavior of the use, how well can we discover and extract the behaviors by analyzing a user logs.
We can note that each of the examples of behaviors given above can be represented by the smatphone logs as a set containing a specific types of realizations. For example, the behavior working on the week days can be represented by the set $\{$ $(day: monday),$ $(day: tuesday),$$...,$$(day: friday),$$(hour: 8am-6pm),$ $(location: work) \}$.  The behavior listening to music while driving could be represented by the set $\{ (Activity: in\- vehicle),$ $(Application Launch: music)\}$.

More generally, most of humans behaviors that can be expressed by smartphone logs that one think of can be represented as a group containing some realizations. For this reason and starting from this observations, the approach that we decide to take in our work is to consider a behavior as a set of data logs realizations.

We define a behavior z as a set of data realizations.  By considering smartphone logs as a data showing a subsample of a user s life, then the task of finding the behaviors that are the most representative of his life is equivalent to find the groups of realizations that are the most descriptive of the smartphone dat. The terms $habbits$, $class$ or $cluster$ are also used to refer to a behavior.

Formulating our problem differently, we aim to find a set of $K$ classes $\{ z_1,...,z_K \} $, where each class is a group of realizations. This means that those classes represents short descriptions of the dataset while preserving the essential statistical relationships of the data. It is this requirement that imposes the classes to represent real user s behaviors.

In section 8, we discuss in more details what does this requirement mean, what does it imply and how it can be measured.
%----------------------------------------------------------------------------------------


%----------------------------------------------------------------------------------------

\section{Related work}
In this section we expose the different researches that have been done using smartphone logs and that are relevant to our work. Smartphone logs are very rich datasets containing multiple types of informations and with an abundant quantity, thus they can be used to achieve different goals and to tackle a large range of different problems. Etter and al. \cite{wheretogo} used smartphone logs to predict the next location to which the user will go. Zhu and al. \cite{mobapp} used them to classify smartphone applications. We detail below two works that are specifically relevant to ours.

Cao and al. \cite{interaction} addressed in 2010 the following problem: having a dataset of smartphone logs, they tried to discover the context that causes a user to have a certain interaction with the phone. An interaction could be $?listening$ $to$ $music?$, $?reading$ $news?$, $?having$ $a$ $message$ $session?$. A context is a set composed by the features that are not phone interactions. For example a Context is $C= \{$$?Is$ $holiday$$=$$Yes?$, $?Day period$$=$$morning?$, $?Phone$ $profile$$=$$silent?$, $?speed$$=$$High?\}$. They were for example interested in learning that the context $C$ usually implies the interaction $?reading$ $news?$.
\\To that end, they used association rules to learn the contextual information that leads to a phone interaction. The idea is to look through all the contextual features, build contexts and count how many times a context appears with a certain type of phone interaction. As checking all the combinations of the contextual features exponentially explode, thresholds $min_support$ and $min_confidence$ are used to only go through the promising contexts.
\\We can note that the problem tackled by \cite{interaction} is similar to our problem in the sense that they tried to learn a user behavior by processing his smartphone logs. However, they were interested in a specific type of behavior, the behavior of a user that leads him to make an interaction with his phone. In our work, we are considering smartphone logs as a sample of a user's life and are interested in discovering the behaviors that drive his life (and not only the ones related to the interaction with his smartphone). To that end we want to stress out that we do not input any prior knowledge or general behavior that humans might have. For example we do not take profit from the fact that most of people have different behaviors during the week days and the week ends (i.e separate the week end from the week day). This is to be able to catch any kind of behavior and in any kind of users. Indeed, if Bob is a farmer, he may want to take a rest of Fridays and not during the week ends.\par

In extension to this work, Ma et al. \cite{superbehaviors} tried to detect similarities between users based on their habits (2012). Based on the same dataset than \cite{interaction} and taking the results of \cite{interaction} as input (i.e having learned context interaction relations for each user individually), they tried to cluster the users by behavior similarities. This work could be used for context-aware recommendation systems. Context aware recommendation systems take into account the context of the user to give him the right recommendation at the right time.
\\To extract super- behaviors (which are clusters of behaviors) of users, they use $Linearly$ $Constrained$ $Bayesian$ $Matrix$ $Factorization$ ($LCBMF$) model described in \cite{superbehaviors}. This method is a bayesian matrix factorization technique that enables to impose some prior linear constrains. $LCBMF$ showed good performances in \cite{superbehaviors}, and noting that our task is similar (extracting clusters from smartphone logs), we apply this method to our problem and discuss it in more details in \ref{4.2}. 

 
% Chapter Template

\chapter{Generative Hidden Class Model for Mixed Data Types (GHCM-MDT)} % Main chapter title

\label{Chapter3} % Change X to a consecutive number; for referencing this chapter elsewhere, use \ref{ChapterX}

\lhead{Chapter 3. \emph{Generative Hidden Class Model for Mixed Data Types (GHCM-MDT)}} % Change X to a consecutive number; this is for the header on each page - perhaps a shortened title

We recall that our task is the following: by observing user logs, we want to discover the behaviors and habits that describe his life. 
\\To that end, we use a usual and common practice when trying to extract some hidden properties (behaviors) from an observable structure (logs): We assume that the logs we are observing are generated by behaviors. Then our task is to find the behaviors that generated the data we are observing. This practice drives the models we are going to discuss in the next sections.

%----------------------------------------------------------------------------------------
%	SECTION 1
%----------------------------------------------------------------------------------------

\section{Hidden Class Model for Mixed Data Types (HCM-MDT)}
The first model we introduce is the Hidden Class Model for Mixed Data Types (HCM-MDT). As the title imply it, this model is used to model hidden (i.e unobserved) classes for a data that contains mixed types. The smartphone logs is a mixed types dataset in the sense that it contains multiple features and the hidden classes that we want to model are the behaviors.
\\This section is organized as follows. First, we describe the HCM-MDT model. Second, to better understand the utility and the intuitions that lead us to build the HCM-MDT model, we make a parallel between this model and the Probabilistic Latent Semantic Indexing (pLsi), which is a model that is widely used to model hidden classes in a corpus of documents. More generally, pLSI can be used to model hidden classes in any kind of dataset that contain a unique type feature (for document corpus dataset, the unique feature is words).
%-----------------------------------
%	SUBSECTION 1
%-----------------------------------
\subsection{HCM-MDT model}

Let's consider the corpus representation of the smartphone logs. Smartphone logs are represented by a corpus  containing $\boldsymbol{R}$ containing $\mathit{M}$ records where each record $\mathbf{r}$ is a vector representing the realizations that occurred in a given time frame $\mathit{T}$ ($\mathit{T}$ could be 1 hour for example).
\\HCM-MDT assumes that each record $\mathbf{r}$ in the corpus $\boldsymbol{R}$ was generated in the following way:


%-----------------------------------
%	SUBSECTION 2
%-----------------------------------
\subsection{Relationship between HCM-MDT and Probabilistic Latent Semantic Indexing (pLSI)}

Nunc posuere quam at lectus tristique eu ultrices augue venenatis. Vestibulum ante ipsum primis in faucibus orci luctus et ultrices posuere cubilia Curae; Aliquam erat volutpat. Vivamus sodales tortor eget quam adipiscing in vulputate ante ullamcorper. Sed eros ante, lacinia et sollicitudin et, aliquam sit amet augue. In hac habitasse platea dictumst.

%----------------------------------------------------------------------------------------
%	SECTION 2
%----------------------------------------------------------------------------------------

\section{Generative HCM-MDT (GHCM-MDT) model}

Sed ullamcorper quam eu nisl interdum at interdum enim egestas. Aliquam placerat justo sed lectus lobortis ut porta nisl porttitor. Vestibulum mi dolor, lacinia molestie gravida at, tempus vitae ligula. Donec eget quam sapien, in viverra eros. Donec pellentesque justo a massa fringilla non vestibulum metus vestibulum. Vestibulum in orci quis felis tempor lacinia. Vivamus ornare ultrices facilisis. Ut hendrerit volutpat vulputate. Morbi condimentum venenatis augue, id porta ipsum vulputate in. Curabitur luctus tempus justo. Vestibulum risus lectus, adipiscing nec condimentum quis, condimentum nec nisl. Aliquam dictum sagittis velit sed iaculis. Morbi tristique augue sit amet nulla pulvinar id facilisis ligula mollis. Nam elit libero, tincidunt ut aliquam at, molestie in quam. Aenean rhoncus vehicula hendrerit.

%----------------------------------------------------------------------------------------
%	SECTION 3
%----------------------------------------------------------------------------------------

\section{GHCM-MDT inference and parameter estimation}

Sed ullamcorper quam eu nisl interdum at interdum enim egestas. Aliquam placerat justo sed lectus lobortis ut porta nisl porttitor. Vestibulum mi dolor, lacinia molestie gravida at, tempus vitae ligula. Donec eget quam sapien, in viverra eros. Donec pellentesque justo a massa fringilla non vestibulum metus vestibulum. Vestibulum in orci quis felis tempor lacinia. Vivamus ornare ultrices facilisis. Ut hendrerit volutpat vulputate. Morbi condimentum venenatis augue, id porta ipsum vulputate in. Curabitur luctus tempus justo. Vestibulum risus lectus, adipiscing nec condimentum quis, condimentum nec nisl. Aliquam dictum sagittis velit sed iaculis. Morbi tristique augue sit amet nulla pulvinar id facilisis ligula mollis. Nam elit libero, tincidunt ut aliquam at, molestie in quam. Aenean rhoncus vehicula hendrerit.

%-----------------------------------
%	SUBSECTION 1
%-----------------------------------
\subsection{Gibbs sampling}

Nunc posuere quam at lectus tristique eu ultrices augue venenatis. Vestibulum ante ipsum primis in faucibus orci luctus et ultrices posuere cubilia Curae; Aliquam erat volutpat. Vivamus sodales tortor eget quam adipiscing in vulputate ante ullamcorper. Sed eros ante, lacinia et sollicitudin et, aliquam sit amet augue. In hac habitasse platea dictumst.

%-----------------------------------
%	SUBSECTION 2
%-----------------------------------
\subsection{Hyperparameters estimation}

Nunc posuere quam at lectus tristique eu ultrices augue venenatis. Vestibulum ante ipsum primis in faucibus orci luctus et ultrices posuere cubilia Curae; Aliquam erat volutpat. Vivamus sodales tortor eget quam adipiscing in vulputate ante ullamcorper. Sed eros ante, lacinia et sollicitudin et, aliquam sit amet augue. In hac habitasse platea dictumst.


%----------------------------------------------------------------------------------------
%	SECTION 4
%----------------------------------------------------------------------------------------

\section{Handling unseen data with GHCM-MDT}

Sed ullamcorper quam eu nisl interdum at interdum enim egestas. Aliquam placerat justo sed lectus lobortis ut porta nisl porttitor. Vestibulum mi dolor, lacinia molestie gravida at, tempus vitae ligula. Donec eget quam sapien, in viverra eros. Donec pellentesque justo a massa fringilla non vestibulum metus vestibulum. Vestibulum in orci quis felis tempor lacinia. Vivamus ornare ultrices facilisis. Ut hendrerit volutpat vulputate. Morbi condimentum venenatis augue, id porta ipsum vulputate in. Curabitur luctus tempus justo. Vestibulum risus lectus, adipiscing nec condimentum quis, condimentum nec nisl. Aliquam dictum sagittis velit sed iaculis. Morbi tristique augue sit amet nulla pulvinar id facilisis ligula mollis. Nam elit libero, tincidunt ut aliquam at, molestie in quam. Aenean rhoncus vehicula hendrerit.



%----------------------------------------------------------------------------------------
%	SECTION 5
%----------------------------------------------------------------------------------------

\section{Relationship between GHCM-MDT and Latent Dirichlet Allocation (LDA)}

Sed ullamcorper quam eu nisl interdum at interdum enim egestas. Aliquam placerat justo sed lectus lobortis ut porta nisl porttitor. Vestibulum mi dolor, lacinia molestie gravida at, tempus vitae ligula. Donec eget quam sapien, in viverra eros. Donec pellentesque justo a massa fringilla non vestibulum metus vestibulum. Vestibulum in orci quis felis tempor lacinia. Vivamus ornare ultrices facilisis. Ut hendrerit volutpat vulputate. Morbi condimentum venenatis augue, id porta ipsum vulputate in. Curabitur luctus tempus justo. Vestibulum risus lectus, adipiscing nec condimentum quis, condimentum nec nisl. Aliquam dictum sagittis velit sed iaculis. Morbi tristique augue sit amet nulla pulvinar id facilisis ligula mollis. Nam elit libero, tincidunt ut aliquam at, molestie in quam. Aenean rhoncus vehicula hendrerit.


 
% Chapter Template

\chapter{Matrix Factorization Models} % Main chapter title

\label{Matrix Factorization Models} % Change X to a consecutive number; for referencing this chapter elsewhere, use \ref{ChapterX}

\lhead{Chapter 4 \emph{Matrix Factorization Models}} % Change X to a consecutive number; this is for the header on each page - perhaps a shortened title

In Chapter 3, we exposed in details the $DLMR$ model and talked about $LMR$, $pLSI$ and $LDA$ which are all based on a probabilistic modeling of the smartphone logs. However, the probabilistic approach in not the only way to tackle the problem. To have a complete overview, we present in this chapter models based on a matrix factorization approach that can extract common latent characteristics (i.e behaviors) from the smartphone logs. First we introduce $Singular$ $Value$ $Decomposition$ ($SVD$) \cite{svd} technique. Then, second we talk about $Linearly$ $Constrained$ $Bayesian$ $Matrix$ $Factorization$ ($LCBMF$) \cite{lcbmf}.
%----------------------------------------------------------------------------------------
%	SECTION 1
%----------------------------------------------------------------------------------------

\section{Singular Value Decomposition (SVD)}

$Singular$ $Value$ $Decomposition$ ($SVD$) \cite{svd} is a common technique for analysis of data and has a wide range of applications. It is for example used in $Latent Semantic Indexing$ ($LSI$) \cite{lsi}, the ancestor of $pLSI$. We briefly present in this section how $SVD$ is used to extract behaviors from smartphone logs.


%-----------------------------------
%	SUBSECTION 1
%-----------------------------------
\subsection{matrix representation of the smartphone logs} \label{4.1.1}

In this section, the corpus of smartphone logs $R$ in represented as a matrix $\boldsymbol{X}$ of $I$ rows and $M$ columns. Each column vector represents a record $\mathbf{r}_{m}$. In this section $\mathbf{r}_{m}$ is a vector of the size of the language defined by $R$, $I=\sum_{f=1}^{J}I_{f}$ where each dimension represents one possible realization.  In this part, each dimension contains the number of times the corresponding realization is observed. \par

$SVD$ finds an approximation $\boldsymbol{\widehat{X}}$ of $\boldsymbol{X}$ that minimizes the Frobenius norm $\left \|  \boldsymbol{X}-\boldsymbol{\widehat{X}}\right \|^{2}$. $\boldsymbol{\widehat{X}}$ is expressed as the product of three matrixes $\boldsymbol{U}$ of dimension $I\times K$, $\boldsymbol{S}$ $K \times K$ and $\boldsymbol{V}$ $\in\mathbb{R}^{K\times M}$, such that $\boldsymbol{S}$ is a diagonal matrix, $\boldsymbol{U}$ and $\boldsymbol{V}$ are orthogonal matrixes.
\\The tree factorizing matrixes can be interpreted as follows. Each of the $K$ columns of $\boldsymbol{U}$ could be interpreted as a behavior, where each dimension of the column vector (of size $I$) contains a value indicating how much the corresponding realization is correlated (positively or negatively) with the given behavior. The diagonal vector of matrix $\boldsymbol{S}$ (of size $K$) contains by definition positive values. They represent the importance of each behavior in the matrix $X$. Finally, each of the $M$ columns of  $\boldsymbol{U}$ could be seen as the behaviors importance in each record, where each dimension of the $m^{th}$ column vector (of size $K$) contains a value indicating how much each behavior is important to describe the record $\mathbf{r}_{m}$.

%-----------------------------------
%	SUBSECTION 2
%-----------------------------------
\subsection{TF-IDF transformation}
As described in \ref{4.1.1} $SVD$ objective function is to minimize the absolute distance $\left \|  \boldsymbol{X}-\boldsymbol{\widehat{X}}\right \|^{2}$. This implies that the dimensions $i\in\{1,...,I\}$ (i.e realizations) that are the most recurrent in $\boldsymbol{X}$ will catch the most the attention of $SVD$ in minimizing its objective function. Thus the most recurrent realizations will get the most importance in the task of finding $\boldsymbol{\widehat{X}}$. While this may seems a good thing at first look (one could argue that the most recurrent realizations should get the most importance because they are the most representative of $X$), it is not true that the most recurrent realizations are the most representative of the data. Indeed, if $Bob$ is always at home and never leaves it, then knowing that Bob is at home does not help in describing current Bob behavior (as it is always the case). Indeed, the very frequent realizations are not carrying meaning and for this reason they should not concentrate all the importance of $SVD$. This problem is also faced in document corpus when using $SVD$ (in $LSI$). \par

To face this problem, we use the $Term$ $Frequency$ $Inverse$ $Document$ $Frequency$ transformation ($TF\_IDF$) \cite{tfidf} on the matrix $\boldsymbol{X}$. This transformation applies a weighting to the different dimensions of $\boldsymbol{X}$ so that it gives more importance to the occurrences that are the most meaningful in describing the data. Each occurrence in $\boldsymbol{X}$ is wighted according to to two criteria: its importance inside the record and its importance inside the corpus. It translates the following intuitions: A realization that is frequent is a record is important in describing this record. A realization that is rare in the corpus carries meaning when it is present in a record.

%----------------------------------------------------------------------------------------
%	SECTION 2
%----------------------------------------------------------------------------------------

\section{Linearly Constrained Bayesian Matrix Factorization (LCBMF)}

While $SVD$ provides nice and easy ways in factorizing matrices and expressing statistical dependence of data, it suffers from non desirable properties. First, $\boldsymbol{U}$ and $\boldsymbol{V}$ contains both positive and negative values which make them complicated and not intuitive to interpret. Second, there is not an objective measurement indicating the quality of the resulting factorization (estimation). $Bayesian$ $Non$ $Negative$ $Matrix$ $Factorization$ \cite{bnmf} came as a response to these problems. In fact, they factorize a matrix $\boldsymbol{X}$ in matrixes containing only positive values and provide uncertainty measures of the factorizations. 
\\$Linearly$ $Constrained$ $Bayesian$ $Matrix$ $Factorization$ ($LCBMF$) is bayesian matrix factorization that provides the possibility to impose any kind of linear constrains to the factorizing matrixes \cite{lcbmf}. It factorizes a matrix $\boldsymbol{X}$ into two matrixes $\boldsymbol{A}$ and $\boldsymbol{B}$ where the elements of $\boldsymbol{A}$ and $\boldsymbol{B}$ are subject to some equality and inequality constrains. To show the benefits and the large degree of freedom that $LCBMF$ brings, we start by exposing some applications of $LCBMF$. Then, we explain the constrains chosen for our problem.
 
%-----------------------------------
%	SUBSECTION 1
%-----------------------------------
\subsection{LCBMF applications and examples}

$LCBMF$ is evaluated in \cite{bnmf} for blind source separation. Handwritten digits grayscale images are superposed randomly to create a dataset of mixed images (each image is a mix of two hand written digits). The task is then to see how well can the sources that generated those images (i.e the handwritten digits) can be recovered. Ideally, one would expect to find nine separate sources where each source represents a different digit. Noting the dataset of mixed images as a matrix $\boldsymbol{X}$ where each column vector represents a mixed image, $\boldsymbol{X}$ $\in\mathbb{R}^{I\times M}$ expressed as the factor of two matrixes $\boldsymbol{A}$ $\in\mathbb{R}^{I\times K}$ and $\boldsymbol{B}$ $\in\mathbb{R}^{K\times M}$ where $K$ is the number of the target hidden sources. By imposing that each column vector in $\boldsymbol{B}$ sums to $1$ and each element in $\boldsymbol{A}$ to be between $0$ and $1$ (note that input grayscale images represent intensities of gray that are between $0$ and $1$), we force the original matrix $\boldsymbol{X}$ to be expressed as the sum of original grayscale sources ($\boldsymbol{A}$) mixed with different proportions (coefficients $\boldsymbol{B}$) (Indeed, each original element $x_{im}$ is expressed as the sum of $x_{im}=\sum_{k=1}^{K}a_{ik}b_{km}$). The linear constrains imposed to $\boldsymbol{A}$ and $\boldsymbol{B}$ allowed $LCBMF$ to adapt to the specific problem of finding handwritten sources digits from mixed images. The results in \cite{bnmf} shows that $LCBMF$ was successful to recover the original digits which is not the case of the other factorizing approaches. \par

in \cite{superbehaviors}, Ma et al. wanted to extract common behaviors of users based on their smartphone logs. They had a matrix $\boldsymbol{X}$ $\in\mathbb{R}^{I\times M}$ where the each row represents a type of behavior and each column represent a user.  The column $m^{th}$ column $\boldsymbol{x}^{m}$ represents the behaviors of the user $m$ where each dimension $i$ contains the number of times the behavior $i$ was observed in the user $m$. To find common users' behaviors (called super-behaviors) from this matrix, they assumed that each user act as a mixture of super-behaviors, where each super behavior is represented as a vector of size $I$ containing the importance of each simple behavior. Thus, they used $LCBMF$ to decompose $\boldsymbol{X}$ into $\boldsymbol{A}$ $\in\mathbb{R}^{I\times M}$ and $\boldsymbol{B}$ $\in\mathbb{R}^{K\times M}$, where  $k^{th}$ column of $\boldsymbol{A}$ represents the super-behavior $k$ and the $m^{th}$ column of $\boldsymbol{B}$ represents the mixture coefficients of the super-behaviors for the user $m$. For this, they imposed that each column vector in $\boldsymbol{B}$ sums to $1$ (representing mixture over super-behaviors) and element in $\boldsymbol{A}$ to be positive. \par

Those two examples show the ability of $LCBMF$ to adapt to different problems and expose the large degree of freedom provided by the ability to fix linear constrains. In the next part, we show the linear constrains that we used to answer our problem.

%-----------------------------------
%	SUBSECTION 2
%-----------------------------------
\subsection{Linear constrains for smartphone logs matrix} \label{4.2}

In this part, we represent smartphone logs as a matrix $\boldsymbol{X}$ of $I$ rows and $M$ columns. As in \ref{4.1.1}, each column vector represents a record $\mathbf{r}_{m}$. Moreover, $\mathbf{r}_{m}$ is also a vector of the size of the language ($I=\sum_{f=1}^{J}I_{f}$) defined by $R$ where each dimension represents one possible realization. However, in this part, each dimension contains the number of observation of a given realization $(f,v)$ normalized by the total number of realizations observed for feature $f$. This means that in each records, the values attributed to the different realizations belonging to the same feature sum to $1$. For example, if Bob opened $2$ times his preferred news application and $1$ time his e-mail application (during the time frame of a record observation) then the realization corresponding to ($"application$ $launch"$, $"news$ $app"$) would get the score of $\frac{2}{3}$ and the realization corresponding to ($"application$ $launch"$, $"e-mail$ $app"$) would get the score of $\frac{1}{3}$. The realizations corresponding to the other application launches would get the score of $0$. By doing this, the records would represent the probabilities of the user's actions. \par

Having $\boldsymbol{X}$, the goal is to extract the user's behaviors. Thus, we use $LCBMF$ and we impose the matrix $\boldsymbol{A}$ $\in\mathbb{R}^{I\times K}$ to represent the behaviors and the matrix $\boldsymbol{B}$ $\in\mathbb{R}^{K\times M}$ to represent the mixture coefficients of the behaviors in each record. We do this by setting each column vector $\mathbf{b}_{m}$ in $\boldsymbol{B}$ sums to $1$ (to represent the coefficients of behaviors in record $m$). Concerning the columns of $\boldsymbol{A}$, we impose the values attributed to the realizations of the same feature to sum to $1$ (the same as the columns of the original matrix). Thus, an observed record $\mathbf{r}_{m}$ would be obtained by the mixture of the different behaviors represented by the columns $\{\mathbf{a}_{k}\}_{k=\{1,...,K\}}$ of matrix $\boldsymbol{A}$. Indeed a record vector $\mathbf{r}_m$ is expressed as by $\mathbf{r}_m = \sum_{k=1}^{K}b_{k,m}\mathbf{a}_{k}$. \par

This concludes the chapter about matrix factorization models. In this chapter we presented two different matrix factorization techniques that are candidate to solve our problem. We talked about $SVD$ a common and widely used matrix factorization technique and $LCBMF$, a sophisticated and powerful technique that we adapted according to our needs.
% Chapter Template

\chapter{Evaluation metrics} % Main chapter title

\label{Chapter5} % Change X to a consecutive number; for referencing this chapter elsewhere, use \ref{ChapterX}

\lhead{Chapter 5. \emph{Evaluation metrics}} % Change X to a consecutive number; this is for the header on each page - perhaps a shortened title
Our goal is to find clusters that represent behaviors of a user from his smartphone logs. Thus, the clusters found by the different models must be really representative of the behavior of the user and not just some random realizations clustered together. For this reason, a way to verify this point is needed. So far, we presented different models that are able to complete the task of finding clusters from smartphone logs. Some of them use a probabilistic approach ($DLMR$, $LMR$, $LDA$, $pLSI$) while the others rely on matrix factorization techniques ($SVD$, $LCBMF$). We develop in this chapter metrics we use to evaluate how much the hidden classes found by a given model are representative of the life of the user. Recalling that smartphone logs are a subsample of a user's life, the common idea that drives those metrics is the following: If the clusters found by analyzing smartphone logs are able to well describe future coming logs (i.e unseen data) from the same user, then those clusters correspond to behaviors that the user use to follow.

%----------------------------------------------------------------------------------------
%	SECTION 2
%----------------------------------------------------------------------------------------
\section{Features prediction}

To see how well the clusters resulting from a model $\mathfrak{M}$ describe an unseen data, the ability of the model to guess the values of a missing feature from a new given record is a good indicator. In other terms, if the realization of a feature $f$ is removed from a new record $\mathbf{r}$, the ability of $\mathfrak{M}$ in guessing this feature by observing the context present in $\mathbf{r}$ indicates how well it generalizes in the data. Indeed, making a good guess means that the clusters produced by $\mathfrak{M}$ were able to represent the context described by a record $\mathbf{r}$ that was not previously seen. For example, if $\mathfrak{M}$ is able to learn that Bob loves listening to music from his smartphone while driving (i.e one of the clusters $z$ of $\mathfrak{M}$ represents the behavior $z="listening\_to\_music\_in\_car"$), and if a record of Bob listening to music (from which the activity of Bob is removed) is given to $\mathfrak{M}$, then $\mathfrak{M}$ would guess that Bob is probably driving (i.e feature $Activity="in_vehicle"$). In the same way, if $\mathfrak{M}$ is able to learn that Bob goes to the gym on Saturdays morning, and if a record showing that Bob is in the Gym is given to $\mathfrak{M}$, then $\mathfrak{M}$ could guess that this record is probably happening a Saturday morning. Because it is a criterion representative of clustering quality, predicting (i.e guessing) missing entries from new data inputs is a common criterion used to compare and evaluate the performance of latent class models. It is used by T. Hofmann [Probabilistic Latent Semantic Indexing]\cite{plsi} to proof that $pLSI$ performs better than it's ancestor $LSI$\cite{lsi} and used again by D. M. Blei, A. Y. Ng and M. I.Jordan [Latent Dirichlet Allocation]\cite{lda} to show that $LDA$ performs better than $pLSI$. It is also used in \cite{inferencecomp, gibbsaverage, gibbsunseen}.\par

Concretely, this is done by splitting the corpus of smartphone logs $R$ in two parts: a train corpus $R_{train}$ and a test corpus $R_{test}$. $R_{train}$ is used to fit the parameters of the model $\mathfrak{M}$ and find the clusters. Then, the ability of $\mathfrak{M}$ to correctly guess missing feature $f$ is evaluated in the records belonging to $R_{test}$  from which the realizations belonging to $f$ are removed. We note those vectors $[ \mathbf{r}_{1}^{-f},...,\mathbf{r}_{M_{test}}^{-f}]$ where $M_{test}$ is the number of records in $R_{test}$.
We spread the values $V_{f}$ in different target categories that we want to guess (for example we divide the dictionary of feature $Location$ in $3$ categories $most\_frequent\_location$, $second\_most\_frequent$, $others$). Then for each record $\mathbf{r}_{m}^{-f}, m \in \{1,...,M_{test}\}$, the model $\mathfrak{M}$ makes a prediction $v_{m}^{pred}=\mathfrak{M}(\mathbf{r}_{m}^{-f}), v_{m}^{pred} \in V_{f}$. We say that $\mathfrak{M}$ made a good guess for $\mathbf{r}_{m}^{-f}$ if $v_{m}^{pred}$ belongs to the right category. 
\\We define $Accuracy_{\mathfrak{M}}$ as the rate of good guesses made by model $\mathfrak{M}$. In other terms,
\begin{equation}
Accuracy_{\mathfrak{M}}=\frac{n_{good\_guesses}}{n_{guesses}}
\end{equation}
where 
\begin{itemize} 
	\item $n_{good\_guesses}$  represents the number of good guesses made by $\mathfrak{M}$.
	\item $n_{guesses}$ represents the attempts made by $\mathfrak{M}$. Note that $n_{tries}=M_{test}$.
\end{itemize}
However, while $Accuracy_{\mathfrak{M}}$ seems to be a nice indicator for clustering and generalization performance, it suffers from a problem when dealing with unbalanced categories' sizes. Indeed, if Bob is $80\%$ of the time at home, and leaves only sometimes his home to go at work or to the gym, then a naive classifier that would always guess the most frequent location of Bob would make a high score of $0.8$ of good guesses. For this reason, we combine $Accuracy_{\mathfrak{M}}$ with another indicator that computes the rate of good predictions for each category separately and average them. We call this indicator $Average\_Accuracy_{\mathfrak{M}}$. Noting $C=\{C_{1},C_{2},...\}$ the set of the different target categories, we can express $Average\_Accuracy_{\mathfrak{M}}$ as follows:
\begin{equation}
Average\_Accuracy_{\mathfrak{M}}=\frac{1}{|C|}\sum_{C_{i}\in C}\frac{n_{good\_guesses,C_{i}}}{n_{(.,C_{i})}}
\end{equation}
where 
\begin{itemize} 
	\item $n_{good\_guesses,C_{i}}$  represents the number of times $\mathfrak{M}$ guessed class $C_{i}$ and the guess was correct.
	\item $n_{(.,C_{i})}$ represents the number of times $\mathfrak{M}$ should have guessed class $C_{i}$. Note that $n_{(.,C_{i})}$ is equal to the number of records containing a value belonging to $C_{i}$ (before removal of 				values).
\end{itemize}
Using the new metric, the naive classifier described above would have a score of $Average\_Accuracy_{naive}=\frac{1}{3}(Accuracy_{Home}+Accuracy_{Work}+Accuracy_{Others})=0.33$, which represents a low score. To evaluate the generalization performances of a model, one should always look at the scores of both $Accuracy$ and $Average\_Accuracy$.\par

To make the guesses $v_{m}^{pred}=\mathfrak{M}(\mathbf{r}_{m}^{-f})$, probabilistic models use maximum likelihood while matrix factorization models use projection space techniques. In the next sections, we explain each of the two techniques. 
%----------------------------------------------------------------------------------------
%	SECTION 2
%----------------------------------------------------------------------------------------
\section{Maximum Likelihood}

We presented in chapter 3 methods that model a corpus of records as a probability distribution and provide ways to compute the likelihood $\mathit{L}$, which is the probability of having some observed samples. In this section, we take profit from this characteristic to make guesses by selecting the most probable guess (i.e the guess that has the maximum likelihood). \par

In $LMR$, the probability of a value $v \in V_{f}$ given a record $\mathbf{r}^{-f}$ is obtained from the posterior distribution over behaviors (i.e clusters):
\begin{equation}
p(v|\mathbf{r}^{-f})=\sum_{k=1}^{K}p(v|Z=z_{k}, F=f)p(Z=z_{k}|\mathbf{r}^{-f})
\end{equation}
$p(v|Z=z_{k}, F=f)$ are the values distributions in behaviors already estimated during training, while $p(Z=z_{k}|\mathbf{r}^{-f})$ is estimated using the $Expectation$ $Maximization$ ($EM$) algorithm described in \cite{plsi}. \par

In $DLMR$ and $LDA$, the probability of a value $v \in V_{f}$ given a record $\mathbf{r}^{-f}$ is obtained by integrating over the posteriors. For $DLMR$ this gives:
\begin{equation}
p(v|\mathbf{r}^{-f})=\int_{\boldsymbol{\theta}_{test} }p(\boldsymbol{\theta}_{test}|\mathbf{r}^{-f},\boldsymbol{\alpha })\sum_{k=1}^{K}p(w_{n}|\boldsymbol{\widehat{\phi}  }_{y_{n},k})p(Z=z_{k}|\boldsymbol{\theta }_{test})
\end{equation}
where $\boldsymbol{\widehat{\Phi }}=\{\boldsymbol{\widehat{\phi} }_{f,k}\}_{\forall f \in F, \forall k\in\{1,...,K\}}$ are the vocabularies distributions estimated on $R_{train}$. As seen in ~\ref{gibbs_sampling}, integrating over $\boldsymbol{\theta}_{test}$ is not possible. Thus, one could estimate a behavior distribution $\boldsymbol{\widehat{\theta}}_{test}$ for record $\mathbf{r}^{-f}$ by proceeding as behaviors distributions are estimated for training set ~\ref{gibbs_sampling}. In our work, we use a variant of Collapsed Gibbs Sampling, a technique developed by Y. Papanikolaou and al. in \cite{gibbsunseen} for treating unseen records. Indeed, it is shown that this estimation technique for unseen records performs better than the traditional ones. This technique was initially developed for $LDA$, but thanks to the similarities between $LDA$ and $DLMR$, it is easily adapted to $DLMR$. Moreover, to obtain the final estimation $\boldsymbol{\widehat{\theta}}_{test}$, we average multiple intermediate samples using the method described in \cite{gibbsaverage}.

Similarly for $LDA$, the probability of a value $v \in V_{f}$ given a record $\mathbf{r}^{-f}$ is expressed as
\begin{equation}
p(v|\mathbf{r}^{-f})=\int_{\boldsymbol{\theta}_{test} }p(\boldsymbol{\theta}_{test}|\mathbf{r}^{-f},\boldsymbol{\alpha })\sum_{k=1}^{K}p((y_{n},w_{n})|\boldsymbol{\widehat{\phi}  }_{k})p(Z=z_{k}|\boldsymbol{\theta }_{test})
\end{equation}
where $\boldsymbol{\widehat{\Phi }}=\{\boldsymbol{\widehat{\phi} }_{k}\}_{\forall k\in\{1,...,K\}}$ are the language distributions estimated on $R_{train}$. Here again, we use \cite{gibbsunseen} and \cite{gibbsaverage} to estimate the behaviors distribution $\boldsymbol{\widehat{\theta}}_{test}$ of the record $\mathbf{r}^{-f}$. \par

We have shown how each of the probabilistic models proceed to attribute likelihood to unobserved values from new records. Then, the predicted value $v$ given by those models is simply the value $v$ belonging to $f$ that has the biggest likelihood. In the next section, we discuss how matrix factorization models proceed to do the same task.

%----------------------------------------------------------------------------------------
%	SECTION 3
%----------------------------------------------------------------------------------------

\section{Space Projection}

We recall that both $SVD$ and $LCBMF$ express the smartphone logs matrix as a product between matrixes. In both models, there is a resulting matrix that represents the behaviors expressed by the data ($\boldsymbol{U}$ for $SVD$, $\boldsymbol{A}$ for $LCBMF$) and another that expresses the concentration of behaviors in each record ($\boldsymbol{V}$ for $SVD$, $\boldsymbol{B}$ for $LCBMF$). Moreover, both $SVD$ and $LCBMF$ represent a record $\mathbf{r}$ as a vector of the size of the language (i.e number of possible realizations) where each dimension contains the number of times a given realization was observed. In this context, a record $\mathbf{r}$ from which all the realizations of the feature $f$ were removed is represented as a vector $\mathbf{r}^{-f}$ that contains $0$ for the realizations of feature $f$ and the same values as $\mathbf{r}$ elsewhere.
\\When having a new record $\mathbf{r}^{-f}$, the idea for the two models is to map $\mathbf{r}^{-f}$ into the space of the transformation (to express the concentration of behaviors in  $\mathbf{r}^{-f}$) and then project it back to the original space (using the realizations importance in the different behaviors) to obtain an approximation $\mathbf{\widehat{r}}$ of $\mathbf{r}$. Intuitively, the values in $\mathbf{\widehat{r}}$ would represent the values that the model would have guessed for each of the possible realizations. Thus, Having $\mathbf{\widehat{r}}$, a natural choice to make is to guess the value $v$ belonging to the feature $f$ that has the highest estimation in the approximation $\mathbf{\widehat{r}}$ (i.e the value that the model would have attributed the maximum number of realizations from all the values of $f$). \par

For $SVD$, we recall that it expresses the data $\boldsymbol{X}$ as $\boldsymbol{X}\simeq \boldsymbol{U}\boldsymbol{S}\boldsymbol{V}$. $\mathbf{\widehat{r}}$ is obtained from $\mathbf{r}^{(-f)}$ by doing the following:
\begin{enumerate} 
	\item compute the $tf\_idf$ of $\mathbf{r}^{(-f)}$, $\mathbf{r}_{tf\_idf}^{(-f)}=tf\_idf(\mathbf{r}^{(-f)})$.
	\item map $\mathbf{r}_{tf\_idf}^{(-f)}$ into matrix $V$ $\boldsymbol{v}_{\boldsymbol{r}_{tf\_idf}}=\boldsymbol{U}^{-1}\boldsymbol{S}^{-1}\boldsymbol{r}_{tf\_idf}^{(-f)}=\boldsymbol{U}^{t}\boldsymbol{S}^{-1}\boldsymbol{r}_{tf\_idf}^{(-f)}		$. Here we use the fact that $\boldsymbol{U}$ is an orthogonal matrix and thus $\boldsymbol{U}^{-1}=\boldsymbol{U}^{t}$. $\boldsymbol{v}_{\boldsymbol{r}_{tf\_idf}}$ is a vector that contains the behaviors concentration of 			record $\mathbf{r}_{tf\_idf}$.
	\item project back $\boldsymbol{v}_{\boldsymbol{r}_{tf\_idf}}$ to the original space to obtain an approximation of $\mathbf{r}_{tf\_idf}$, $\mathbf{\widehat{r}}_{tf\_idf} = \boldsymbol{U}\boldsymbol{S}\boldsymbol{v}_{\boldsymbol{r}_{tf		\_idf}}$.
	\item do the inverse $tf\_idf$ transformation to obtain an approximation of $\mathbf{r}$, $\mathbf{\widehat{r}} = tf\_idf^{-1}(\mathbf{\widehat{r}}_{tf\_idf})$	
\end{enumerate} \par

Concerning $LCBMF$ it expresses the data $\boldsymbol{X}$ as $\boldsymbol{X}\simeq AB$. $\mathbf{\widehat{r}}$ is obtained from $\mathbf{r}^{-f}$ by doing the following:
\begin{enumerate} 
	\item map $\mathbf{r}$ into matrix $\boldsymbol{B}$ $\boldsymbol{b}_{\boldsymbol{r}}=\boldsymbol{A}^{-1}\boldsymbol{r}^{(-f)}$. Note that matrix $\boldsymbol{A}$ is not invertible by definiton, however there exists methods to 			compute the pseudo inverse at 	the right of a non invertible matrix. $\boldsymbol{b}_{\boldsymbol{r}}$ is a vector that contains the behaviors concentration of record $\mathbf{r}$.
	\item project back $\boldsymbol{b}_{\boldsymbol{r}}$ to the original space to obtain an approximation of $\mathbf{\widehat{r}} = \boldsymbol{A}\boldsymbol{b}_{r}$.
\end{enumerate}
	


% Chapter Template

\chapter{Results and Discussion} % Main chapter title

\label{Chapter6} % Change X to a consecutive number; for referencing this chapter elsewhere, use \ref{ChapterX}

\lhead{Chapter 6. \emph{Results and Discussion}} % Change X to a consecutive number; this is for the header on each page - perhaps a shortened title

%----------------------------------------------------------------------------------------
%	SECTION 1
%----------------------------------------------------------------------------------------
So far, we have developed an advanced probabilistic model that is able to infer user's behavior from smartphone logs. We exposed in details its nice properties and the theoretical advantages it has with respect to the other probabilistics models. To evaluate its performance in completing this task, we decide to confront it to state of the art current methods for doing the same job. To have a rigorous overview, we considered different methods relying on different approaches. In particular, we considered an advanced new approach of matrix factorization that was used by recent researches ($LCBMF$) and that was shown to perform well. Moreover, we considered the state of the art probabilistic approach in hidden class modeling ($LDA$) that showed impressive results since years in representing latent clusters from an observable data.

\noindent
We also discussed in details meaningfull metrics that can be used to objectively compare the performance of these models.

In this chapter, we introduce the dataset used to test these models. Then, we present the results obtained by the different models($DLMR$, $LMR$, $LCBMF$, $SVD$).

%----------------------------------------------------------------------------------------
%	SECTION 1
%----------------------------------------------------------------------------------------

\section{Presenting the Dataset}

We test our models on smartphone logs of five different users. Those smartphone logs belong to five Sony employees in Tokyo and were recorded using an internal Sony software.

\noindent
Those logs contain the records of the users during several months of observation (seven months for the newest user and ten months for the oldest). Each record represents one hour of the period of the observation. The table~\ref{duration} presents the duration of the observation period of the different users.

\begin{table}[H] 
\centering
\begin{tabular}{|l|c|c|c|c|c|}
  \hline
  &User 1 & User 2 & User 3 & User 4 & User 5 \\
 \hline
  Period & 300 & 231 & 229 & 249 &  224 \\
 \hline
\end{tabular}
\caption {Period of observation for each user in days.} 
\label{duration}
\end{table}

Those logs contain the following features :

\begin{enumerate}
\item $Activity$ that represents the activity the user is doing. It takes the values $in\_bicycle$, $in\_vehicle$, $on\_foot$, $tilting$ and $still$.
\item $Application Launch$ that represents the smartphone apps launched by the users.
\item $Bluetooth paired$ represents the bluetooth devices paired with the smartphone of the user.
\item $Day$ represents the day of the week of the record.
\item $Hour$ represents the time of the day of the record.
\item $Location$ represents the location of the users.
\item $Notification$ represents the notifications received by the user.
\end{enumerate}
%----------------------------------------------------------------------------------------
%	SECTION 2
%----------------------------------------------------------------------------------------

\section{Feature prediction results}
To test the different model, we divide the dataset into 80 \% for training and 20 \% for testing. We choose to make predictions on the $Location$ feature. Indeed, one can argue that many individual behaviors of a human are correlated to his location. For this reason, choosing the $Location$ feature as a target feature seems to be a natural choice. This tests the ability of the different models to guess the location in which the user is from the context described by his activity, the application launched, the notifications received, the day of the week, the time of the day and bluetooth devices paired to the smartphone. In this test target categories are $\{ $$most$ $frequent$ $location,$ $second$ $most$ $frequent$ $location$$,$ $other$$\}$. 
Note that for most users, $Home$ is the most frequent location and $Work$ the second most frequent location or the opposite.
\\For each model, we run the prediction tests for the 5 users and then we average the prediction accuracies results (accuracy and average accuracy) between the 5 users. For each user, the predicted location is a good prediction if it belongs to the right category (and bad prediction otherwise).

Figures ~\ref{acc} and ~\ref{avacc} shows, respectively, the accuracy and average accuracy prediction results of the five models ($DLMR$, $LMR$, $LDA$, $LCBMF$ and $SVD$) tested for different number of behaviors (i.e hidden classes).

\begin{figure} [!ht]
\centering
\includegraphics[scale=0.3]{Figures/location_accuracy.png}
\caption{Accuracy results of the prediction of the five models.}
\label{acc}
\end{figure}

\begin{figure} [!ht]
\centering
\includegraphics[scale=0.3]{Figures/location_average_accuracy.png}
\caption{Average accuracy results of the prediction of the five models.}
\label{avacc}
\end{figure}

First, we can see that the three probabilistic models, $DLMR$, $LDA$ and $LMR$ perform widely better than the matrix factorization models $LCBMF$ and $SVD$.
\\Indeed, when the number of behaviors increases ($K\geqslant 10$) the performance of $SVD$ and $LCBMF$ decreases heavily whereas $DLMR$ $LDA$ and $LMR$ stay relatively stable. This might be due to an effect of overfitting. \par

Deepen our observation int the three probabilistic models, we can see that $DLMR$ largely outperforms $LMR$. Indeed, the prediction scores of $DLMR$ are about 10\% better than $LMR$ for both average accuracy and accuracy for all the number of behaviors K. This confirms the intuition developed in chapter 3. While $LMR$ learns the behaviors expressed specifically by the records seen, $DLMR$ tries to catch a more general structure of the corpus by learning how the seen records generate the behaviors. This allows $DLMR$ to describe better an unseen data. Moreover, while both models show good stability to high number of behaviors, we can observe that for $K\geqslant 50$, $LMR$ score start to decrease while $DLMR$ is still enough stable for  $K=70$. The average accuracy plot shows this effect (average accuarcy of $LMR$ decreased by 3\% between $K = 50$ and $K = 70$ while average accuracy of $DLMR$ decreased only by 1 \% between $K = 50$ and $K = 70$, which is more due to the randomness of the test and train set rather than to overfitting). The same effect is present in the accuracies plots. 
\\As explained in chapter 3, $DLMR$ has $K+I$ ($I$ is the size of the language and equals $\sum_{f=1}^{J} I_f$) parameters to estimate while $LMR$ has $KM + KV$ parameters. Thus, when K increases the number of parameters to estimate increases much faster in $LMR$ than in $DLMR$. This is what might explain the observed effect in the sense that the big number of parameters in $LMR$ caused overfitting. \par

Concerning $LDA$, we see that it is also strong to overfitting (and stronger than $LMR$) and accurate in predictions. In fact, $LDA$ underlies the same properties than $DLMR$ (learn how records generate behaviors, small number of parameters to estimate). This is what explains its performance. However, $DLMR$ performs better than $LDA$ in prediction scores. Indeed, while they perform very close, $DLMR$ performs better than $LDA$ in average accuracy while keeping a similar score (or even slightly better score) to $LDA$ in $accuracy$. This means that $DLMR$ is able to better guess less frequent categories without loosing in general accuracy. This implies that $DLMR$ is better in detecting the rare events or contexts. This is explained by the structure of the model of $DLMR$. $DLMR$ imposes the probabilities of the values belonging to the same feature to sum to 1. This represents a much more realistic approximation of the real life than $LDA$ that spreads the probabilities over the space of all possible realizations. While frequent contexts (or events) can be outlined by an acceptable model of the real life, detecting rare contexts requires a much more precise representation.
\\Finally, evaluating $DLMR$ performance independently, we see that it performs around 74\% of good guesses and 67\% of average good guesses per class. This is a good prediction performance that shows that $DLMR$ is able to predict the future actions that the user might take. This implies that $DLMR$ was able to represent and learn the general behaviors according to which a user behaves.

As a sum up, the results show that the probabilistic models developed perform better in guessing right behaviors of users than the matrix factorization models exposed. Comparing the three probabilistic models $DLMR$, $LMR$ and $LDA$, $DLMR$ shows better results than both $LMR$ and $LDA$. This confirms the intuitions developed in chapter 3 that led us to the $DLMR$ model. Finally, the accuracies rate of $DLMR$ shows good generalization performance of unseen data, and thus proves that $DLMR$ is able to discover user behaviors from their logs.


%----------------------------------------------------------------------------------------
%	SECTION 3
%----------------------------------------------------------------------------------------

\section{Feature prediction results}
We recall that the final goal that drives our work is to discover the particular behaviors of a user from his smartphone logs. In this concluding section, we close the circle by showing examples of behaviors of the $5$ users discovered by $DLMR$. \par

First of all, and as one might expect, the $5$ users exhibit behavior that are similar. Indeed, there is at least one behavior for each user representing the fact that the user works in the week days during the day. There is also others expressing the fact that users are more often at home during the weekends, and others indicating that the users are almost always at home during the night (for all the days). An example of each of these $3$ behaviors is shown for one of the users in figure~\ref{commonBehavior} . 
\\However, one could have supposed those behaviors without analyzing smartphone logs. For this reason, they are not the most interesting habits that we aim to discover. We present them here for two reasons. First, it is a way to verify that $DLMR$ was able to discover some a-priori expected behaviors (as if we label users with some behaviors and then check that $DLMR$ is able to discover them). Second, we present them to give an intuition on how exactly the behaviors are represented in $DLMR$. \par

Now, we pay attention to examples of behaviors that are more interesting in the sense that they describe a specific user's habit. For example, the behavior in figure~\ref{readSunday} shows that $user1$ likes to do some readings on Sundays night. Indeed, $Reader$ (\href{https://play.google.com/store/apps/details?id=com.sony.drbd.reader.ext.pictorial.ja&hl=fr}{link}) is the package name of a Japanese smartphone application for purchasing and reading books. Alternatively, $user1$ would play $monster$ $strike$ (\href{https://play.google.com/store/apps/details?id=jp.co.mixi.monsterstrikeUS&hl=fr}{link}) or browse in the web using his smartphone ($Boat$ $Browser$ is the package name of a web browser (\href{https://play.google.com/store/apps/details?id=com.boatbrowser.free&hl=fr}{link})). 
\\The behavior in figure~\ref{news} shows that when the $user4$ is not at work during the day (in the week days), he would probably reads some news or watch some TV program. Indeed $SmartNews$ (\href{https://play.google.com/store/apps/details?id=jp.gocro.smartnews.android}{link}), $Socialife News$ (\href{https://play.google.com/store/apps/details?id=com.sony.nfx.app.sfrc&hl=fr}{link}) and $Gunosy$ (\href{https://play.google.com/store/apps/details?id=com.gunosy.android.world}{link}) are all news applications while $TV SideView Sony$ (\href{https://play.google.com/store/apps/details?id=com.sony.tvsideview.phone&hl=fr}{link}) is a smartphone TV app.
\\Finally, we end our example review with the behavior in figure~\ref{ing} . It shows that $user2$ is often playing ingress while being in $other$ $places$ during the weekends (in the day). In fact, $Ingress$ (l\href{https://play.google.com/store/apps/details?id=com.nianticproject.ingress&hl=fr}{link}) is an augmented reality playing location based game. In other terms, ingress is a game that transforms the real world in a game map, where players go from one place to the other trying to solve challenges. That's what explains the behavior of $user2$. \par

All these examples show that $DLMR$ is able to discover user's behaviors from his smartphone logs. In other terms, it is able to decompose a user's life (smartphone logs are a subsample of user's life) as a set of behaviors that are driving his life. Moreover, it is able to do so by extracting both general behaviors (as sleeping at home, working in the day) and more specific habits (reading on Sundays night, playing ingress on the week end). \par

Let's take a step back and recall the initial wish that led us to try to discover users' habbits. We wanted to enable smartphones to build a personal relationship with their owner by learning to know their habits and reacting to their specific needs. Thus we conclude this part by showing how the examples discussed above could be turned out to that end. For example, the smartphone of $user1$ could suggests to him new trendy books to read on Sundays night (even if $user1$ was not planning to read, maybe because he is bored with the books he is currently reading), or even proposes new games that he could play. Concerning $user2$, if his smartphone knows that it will be rainy next Sunday, then it could alert it's owner and suggests him to schedule his ingress session in another day if he is planning to play. Those are only few examples of possible applications and are far from being exhaustive. The work developed in this thesis does not already enable smartphones to act as described (smartphones need to be able to interpret the behaviors they extracted to do so), but allowing them to catch users' behaviors is a necessary and important step towards that direction.

\begin{figure*}[t!]
    \centering
    \begin{subfigure}[t]{0.33\textwidth}
        \centering
        \includegraphics[scale=0.5]{Figures/Home.png}
        \caption{Home\_night.}
    \end{subfigure}%
    \begin{subfigure}[t]{0.33\textwidth}
        \centering
        \includegraphics[scale=0.5]{Figures/Work.png}
        \caption{Working\_week\_days.}
    \end{subfigure}%
  \begin{subfigure}[t]{0.33\textwidth}
        \centering
        \includegraphics[scale=0.515]{Figures/Homeweekend.png}
        \caption{Home\_weekend.}
    \end{subfigure}

    \caption{Examples of common behaviors.}
\label{commonBehavior}
\end{figure*}


\begin{figure*}[t!]
    \centering
    \begin{subfigure}[t]{0.33\textwidth}
        \centering
        \includegraphics[scale=0.51]{Figures/Ingress.png}
        \caption{play\_ingress\_weekend.}
\label{ing}
    \end{subfigure}%
    \begin{subfigure}[t]{0.33\textwidth}
        \centering
        \includegraphics[scale=0.51]{Figures/News.png}
        \caption{Reading\_news\_mornings.}
\label{news}
    \end{subfigure}%
  \begin{subfigure}[t]{0.33\textwidth}
        \centering
        \includegraphics[scale=0.5]{Figures/sundayReading.png}
        \caption{reading\_sunday\_night.}
\label{readSunday}
    \end{subfigure}
    \caption{Examples of special behaviors.}
\label{specialBehavior}

\end{figure*}

 
% Chapter Template

\chapter{Conclusion} % Main chapter title

\label{Chapter7} % Change X to a consecutive number; for referencing this chapter elsewhere, use \ref{ChapterX}

\lhead{Chapter 7. \emph{Conclusion}} % Change X to a consecutive number; this is for the header on each page - perhaps a shortened title

In this work, our goal was to develop a model that enables extracting user's behaviors and habits from his smartphone logs (and to see how well this can be done).
\\To that end we developed a model that can specifically fit a multimodal data (i.e a data that contains multiple types) by treating each type separately and in the same time combining the different types to discover common relations expressed by the data. We called this model the $Dirichlet$ $Latent$ $Multimodal$ $Representation$ ($DLMR$).
\\We compared $DLMR$ to a complete overview of state of the art methods that could accomplish the same task. In this sense, we both considered matrix factorization approaches and probabilistic modeling approaches.
\\The results obtained show that $DLMR$ performs better than these models and in particular better than $Latent$ $Dirichlet$ $Allocation$ ($LDA$) and $Linearly$ $Constrained$ $Bayesian$ $Matrix$ $Factorization$ ($LCBMF$). Moreover, $DLMR$ shows good performances in discovering both general habits and specific behaviors of users based on their smartphones logs. \par

During this work, the need of developing $DLMR$ came from the specificity of the dataset we are considering. Indeed, while existing models are built to fit homogenous datasets containing only one type of information, $DLMR$ came from the need of modeling the heterogeneity of the smartphone logs dataset.
\\However, smartphone logs is not the only kind of dataset that has these characteristics. On the contrary, with the exponential increase of applications and platforms accessing to web, the rise of objects connected to the internet and the rapid growth of wearable devices (smart watch, smart glasses) and mobile mini-computers, datasets contains more and more multiple types of data. Because it was built by essence to model the multimodality of data, $DLMR$ can be applied to all these kind of datasets because and can be used for different purposes. It can for example be used to give a meaning to a non intuitive dataset, to predict future events, to classify data elements or to compress the size of a large dataset.

 
%\input{Chapters/Chapter6} 
%\input{Chapters/Chapter7} 

%----------------------------------------------------------------------------------------
%	THESIS CONTENT - APPENDICES
%----------------------------------------------------------------------------------------

\addtocontents{toc}{\vspace{2em}} % Add a gap in the Contents, for aesthetics

\appendix % Cue to tell LaTeX that the following 'chapters' are Appendices

% Include the appendices of the thesis as separate files from the Appendices folder
% Uncomment the lines as you write the Appendices

\input{Appendices/AppendixA}
%\input{Appendices/AppendixB}
%\input{Appendices/AppendixC}

\addtocontents{toc}{\vspace{2em}} % Add a gap in the Contents, for aesthetics

\backmatter

%----------------------------------------------------------------------------------------
%	BIBLIOGRAPHY
%----------------------------------------------------------------------------------------

\label{Bibliography}

\lhead{\emph{Bibliography}} % Change the page header to say "Bibliography"

\bibliographystyle{unsrtnat} % Use the "unsrtnat" BibTeX style for formatting the Bibliography

\bibliography{./Bibliography.bib} % The references (bibliography) information are stored in the file named "Bibliography.bib"

\end{document}  