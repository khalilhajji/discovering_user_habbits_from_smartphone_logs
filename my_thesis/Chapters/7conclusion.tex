% Chapter Template

\chapter{Conclusion} % Main chapter title

\label{Chapter7} % Change X to a consecutive number; for referencing this chapter elsewhere, use \ref{ChapterX}

\lhead{Chapter 7. \emph{Conclusion}} % Change X to a consecutive number; this is for the header on each page - perhaps a shortened title

In this work, our goal was to develop a model that enables extracting user's behaviors and habits from his smartphone logs (and to see how well this can be done).
\\To that end we developed a model that can specifically fit a multimodal data (i.e a data that contains multiple types) by treating each type separately and in the same time combining the different types to discover common relations expressed by the data. We called this model the $Dirichlet$ $Latent$ $Multimodal$ $Representation$ ($DLMR$).
\\We compared $DLMR$ to a complete overview of state of the art methods that could accomplish the same task. In this sense, we both considered matrix factorization approaches and probabilistic modeling approaches.
\\The results obtained show that $DLMR$ performs better than these models and in particular better than $Latent$ $Dirichlet$ $Allocation$ ($LDA$) and $Linearly$ $Constrained$ $Bayesian$ $Matrix$ $Factorization$ ($LCBMF$). Moreover, $DLMR$ shows good performances in discovering both general habits and specific behaviors of users based on their smartphones logs. \par

During this work, the need of developing $DLMR$ came from the specificity of the dataset we are considering. Indeed, while existing models are built to fit homogenous datasets containing only one type of information, $DLMR$ came from the need of modeling the heterogeneity of the smartphone logs dataset.
\\However, smartphone logs is not the only kind of dataset that has these characteristics. On the contrary, with the exponential increase of applications and platforms accessing to web, the rise of objects connected to the internet and the rapid growth of wearable devices (smart watch, smart glasses) and mobile mini-computers, datasets contains more and more multiple types of data. Because it was built by essence to model the multimodality of data, $DLMR$ can be applied to all these kind of datasets because and can be used for different purposes. It can for example be used to give a meaning to a non intuitive dataset, to predict future events, to classify data elements or to compress the size of a large dataset.

