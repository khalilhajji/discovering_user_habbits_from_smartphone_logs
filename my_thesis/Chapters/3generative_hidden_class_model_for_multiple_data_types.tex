% Chapter Template

\chapter{Generative Hidden Class Model for Mixed Data Types (GHCM-MDT)} % Main chapter title

\label{Chapter3} % Change X to a consecutive number; for referencing this chapter elsewhere, use \ref{ChapterX}

\lhead{Chapter 3. \emph{Generative Hidden Class Model for Mixed Data Types (GHCM-MDT)}} % Change X to a consecutive number; this is for the header on each page - perhaps a shortened title

We recall that our task is the following: by observing user logs, we want to discover the behaviors and habits that describe his life. 
\\To that end, we use a usual and common practice when trying to extract some hidden properties (behaviors) from an observable structure (logs): We assume that the logs we are observing are generated by behaviors. Then our task is to find the behaviors that generated the data we are observing. This practice drives the models we are going to discuss in the next sections.

%----------------------------------------------------------------------------------------
%	SECTION 1
%----------------------------------------------------------------------------------------

\section{Hidden Class Model for Mixed Data Types (HCM-MDT)}
The first model we introduce is the Hidden Class Model for Mixed Data Types (HCM-MDT). As the title imply it, this model is used to model hidden (i.e unobserved) classes for a data that contains mixed types. The smartphone logs is a mixed types dataset in the sense that it contains multiple features and the hidden classes that we want to model are the behaviors.
\\This section is organized as follows. First, we describe the HCM-MDT model. Second, to better understand the utility and the intuitions that lead us to build the HCM-MDT model, we make a parallel between this model and the Probabilistic Latent Semantic Indexing (pLsi), which is a model that is widely used to model hidden classes in a corpus of documents. More generally, pLSI can be used to model hidden classes in any kind of dataset that contain a unique type feature (for document corpus dataset, the unique feature is words).
%-----------------------------------
%	SUBSECTION 1
%-----------------------------------
\subsection{HCM-MDT model}

Let's consider the corpus representation of the smartphone logs. Smartphone logs are represented by a corpus  containing $\boldsymbol{R}$ containing $\mathit{M}$ records where each record $\mathbf{r}$ is a vector representing the realizations that occurred in a given time frame $\mathit{T}$ ($\mathit{T}$ could be 1 hour for example).
\\HCM-MDT assumes that each record $\mathbf{r}$ in the corpus $\boldsymbol{R}$ was generated in the following way:


%-----------------------------------
%	SUBSECTION 2
%-----------------------------------
\subsection{Relationship between HCM-MDT and Probabilistic Latent Semantic Indexing (pLSI)}

Nunc posuere quam at lectus tristique eu ultrices augue venenatis. Vestibulum ante ipsum primis in faucibus orci luctus et ultrices posuere cubilia Curae; Aliquam erat volutpat. Vivamus sodales tortor eget quam adipiscing in vulputate ante ullamcorper. Sed eros ante, lacinia et sollicitudin et, aliquam sit amet augue. In hac habitasse platea dictumst.

%----------------------------------------------------------------------------------------
%	SECTION 2
%----------------------------------------------------------------------------------------

\section{Generative HCM-MDT (GHCM-MDT) model}

Sed ullamcorper quam eu nisl interdum at interdum enim egestas. Aliquam placerat justo sed lectus lobortis ut porta nisl porttitor. Vestibulum mi dolor, lacinia molestie gravida at, tempus vitae ligula. Donec eget quam sapien, in viverra eros. Donec pellentesque justo a massa fringilla non vestibulum metus vestibulum. Vestibulum in orci quis felis tempor lacinia. Vivamus ornare ultrices facilisis. Ut hendrerit volutpat vulputate. Morbi condimentum venenatis augue, id porta ipsum vulputate in. Curabitur luctus tempus justo. Vestibulum risus lectus, adipiscing nec condimentum quis, condimentum nec nisl. Aliquam dictum sagittis velit sed iaculis. Morbi tristique augue sit amet nulla pulvinar id facilisis ligula mollis. Nam elit libero, tincidunt ut aliquam at, molestie in quam. Aenean rhoncus vehicula hendrerit.

%----------------------------------------------------------------------------------------
%	SECTION 3
%----------------------------------------------------------------------------------------

\section{GHCM-MDT inference and parameter estimation}

Sed ullamcorper quam eu nisl interdum at interdum enim egestas. Aliquam placerat justo sed lectus lobortis ut porta nisl porttitor. Vestibulum mi dolor, lacinia molestie gravida at, tempus vitae ligula. Donec eget quam sapien, in viverra eros. Donec pellentesque justo a massa fringilla non vestibulum metus vestibulum. Vestibulum in orci quis felis tempor lacinia. Vivamus ornare ultrices facilisis. Ut hendrerit volutpat vulputate. Morbi condimentum venenatis augue, id porta ipsum vulputate in. Curabitur luctus tempus justo. Vestibulum risus lectus, adipiscing nec condimentum quis, condimentum nec nisl. Aliquam dictum sagittis velit sed iaculis. Morbi tristique augue sit amet nulla pulvinar id facilisis ligula mollis. Nam elit libero, tincidunt ut aliquam at, molestie in quam. Aenean rhoncus vehicula hendrerit.

%-----------------------------------
%	SUBSECTION 1
%-----------------------------------
\subsection{Gibbs sampling}

Nunc posuere quam at lectus tristique eu ultrices augue venenatis. Vestibulum ante ipsum primis in faucibus orci luctus et ultrices posuere cubilia Curae; Aliquam erat volutpat. Vivamus sodales tortor eget quam adipiscing in vulputate ante ullamcorper. Sed eros ante, lacinia et sollicitudin et, aliquam sit amet augue. In hac habitasse platea dictumst.

%-----------------------------------
%	SUBSECTION 2
%-----------------------------------
\subsection{Hyperparameters estimation}

Nunc posuere quam at lectus tristique eu ultrices augue venenatis. Vestibulum ante ipsum primis in faucibus orci luctus et ultrices posuere cubilia Curae; Aliquam erat volutpat. Vivamus sodales tortor eget quam adipiscing in vulputate ante ullamcorper. Sed eros ante, lacinia et sollicitudin et, aliquam sit amet augue. In hac habitasse platea dictumst.


%----------------------------------------------------------------------------------------
%	SECTION 4
%----------------------------------------------------------------------------------------

\section{Handling unseen data with GHCM-MDT}

Sed ullamcorper quam eu nisl interdum at interdum enim egestas. Aliquam placerat justo sed lectus lobortis ut porta nisl porttitor. Vestibulum mi dolor, lacinia molestie gravida at, tempus vitae ligula. Donec eget quam sapien, in viverra eros. Donec pellentesque justo a massa fringilla non vestibulum metus vestibulum. Vestibulum in orci quis felis tempor lacinia. Vivamus ornare ultrices facilisis. Ut hendrerit volutpat vulputate. Morbi condimentum venenatis augue, id porta ipsum vulputate in. Curabitur luctus tempus justo. Vestibulum risus lectus, adipiscing nec condimentum quis, condimentum nec nisl. Aliquam dictum sagittis velit sed iaculis. Morbi tristique augue sit amet nulla pulvinar id facilisis ligula mollis. Nam elit libero, tincidunt ut aliquam at, molestie in quam. Aenean rhoncus vehicula hendrerit.



%----------------------------------------------------------------------------------------
%	SECTION 5
%----------------------------------------------------------------------------------------

\section{Relationship between GHCM-MDT and Latent Dirichlet Allocation (LDA)}

Sed ullamcorper quam eu nisl interdum at interdum enim egestas. Aliquam placerat justo sed lectus lobortis ut porta nisl porttitor. Vestibulum mi dolor, lacinia molestie gravida at, tempus vitae ligula. Donec eget quam sapien, in viverra eros. Donec pellentesque justo a massa fringilla non vestibulum metus vestibulum. Vestibulum in orci quis felis tempor lacinia. Vivamus ornare ultrices facilisis. Ut hendrerit volutpat vulputate. Morbi condimentum venenatis augue, id porta ipsum vulputate in. Curabitur luctus tempus justo. Vestibulum risus lectus, adipiscing nec condimentum quis, condimentum nec nisl. Aliquam dictum sagittis velit sed iaculis. Morbi tristique augue sit amet nulla pulvinar id facilisis ligula mollis. Nam elit libero, tincidunt ut aliquam at, molestie in quam. Aenean rhoncus vehicula hendrerit.


